\documentclass[12pt]{article}
\pdfminorversion=7

% Standard packages for figures, citations, math, tables, etc.
% Feel free to add more as needed!
\usepackage{amssymb}
\usepackage{graphicx}
\usepackage{cite}
\usepackage[cmex10]{amsmath}
\interdisplaylinepenalty=2500 
\usepackage{mdwmath}
\usepackage{mdwtab}
\usepackage{color}
\usepackage{pgfplots}
\usepackage{multirow}
\usepackage{subcaption}
\usepackage{verbatim}
\usepackage{enumitem}
\usepackage{appendix}
\usepackage[outdir=./]{epstopdf}
\usepackage{indentfirst}
\pgfplotsset{compat=1.17}

% Packages for formatting with 1 in margins
\usepackage[latin1]{inputenc}
\usepackage[left=1in,top=1in,right=1in,bottom=1in,nohead]{geometry}
\usepackage{setspace}
\setstretch{1.15}
\setlength\parindent{1cm}

% Define the location of your figures in order to avoid writing absolute file location
\graphicspath{{figures/}}	

\begin{document}

% Title Page
\begin{titlepage}
\begin{center}
{\LARGE \textsc{ECE 445: Senior Design Laboratory} \\ \vspace{8pt}}
\rule[13pt]{\textwidth}{1pt} \\ \vspace{120pt}
{\huge \textbf{\textsc{The Odds Booster}} \\ \vspace{8pt}}
{\LARGE \textbf{\textsc{Design Document}} \\ \vspace{30pt}} 
{\large \textit{Authors:} Marco Rojas \\ \vspace{4pt}
\hspace{48pt} Jack Arndt \\ \vspace{4pt}
\hspace{48pt} Tim Green \\ \vspace{4pt}
\hspace{8pt} \textit{Date Written:} February 20th, 2022}
\vfill
\end{center}

% Numbered pages on everything but title
\pagenumbering{arabic}	
\end{titlepage}
\setcounter{page}{2}

% Introduction can be kept the same as the proposal (same idea from rubric)
\section{Introduction}

Before heading to the casino, individuals need to assess how much money they are willing to lose, since, as the saying goes, ``The house always wins''. Or do they? Is there potentially a way to bring the casino games to the comfort of your home and train/optimize your strategies to ask a different question, ``How much money should I win?''

The Odds Booster is an automated casino game assistant designed to help players learn and manage casino games such as Blackjack or Texas Hold'em. The kit involves a central communicable device along with a custom deck of cards which will be used to determine the best possible move for each player along with the outcome of each hand of the game. This device is an innovation as it brings both the ease and simplicity as well as the ability to learn the game that can be provided by a virtual game into the superior enjoyment and atmosphere of a physical game. Digital poker tables that provide a similar experience exist but cost thousands of dollars. Our design achieves this functionality without an expensive custom table and while allowing for physical cards and chips.

The Odds Booster central device will serve as the hub used by the dealer and all players of the game. The dealer will swipe each card being distributed to the players over the central device and said device will determine the specific card being dealt using an RFID sensor and thin passive RFID tags placed on the cards. It may be asked, ``Why the more expensive RFID tags when a camera could be used to detect the card?'' Implementing a camera to read a QR code or implementing computer vision to ``see'' the card being swiped requires more specific card orientation, takes more time, and requires a lighting source. All these are not ideal for a dealer trying to swipe and pass out cards quickly. This central device will also contain a display detailing the next actions and outcomes for each hand of the game helping each player and the dealer learn the rules of the game. This main device will be paired via Bluetooth to a user's phone app. The card information will be shared with the app and the app will tell the player the strength of their hand or what move will have the best outcome within the context of the game. It is important to note that knowledge of the hands of the other players within the game provides an unfair advantage to the user and such information will be ignored when determining moves for the user. The ethics of the game will be discussed further later in this proposal. The device itself will be battery powered and rechargeable for easy use and mobility. The usage of the device is portrayed in the diagram shown in Figure \ref{fig:use_dia}.

\begin{figure}[!h]
	\centering
	\includegraphics[width=0.9\textwidth]{ProposalUsageDiagram.png}
	\caption{System Usage Diagram}
	\label{fig:use_dia}
\end{figure}

\noindent
The main requirements to ensure the completion and success of this project are as follows.
\begin{itemize}
\item The central device must be able to sense and distinguish the 52 different RFID tags associated with each card.
\item The central device must be able to display the card and game information for all players to see as well as send the information to a phone app via Bluetooth.
\item The phone app must be able to receive game information via Bluetooth and use the information to determine the strength of the user's hand and/or the best possible move for the user.
\end{itemize}
These requirements are broken into more detailed and subsystem specific requirements through the remainder of this proposal.

% Design parts have same structure
\section{Design}

The block diagram describing each subsystem within the system and their connections is shown in Figure \ref{fig:block}. The design is composed of five main subsystems. Four of the subsystems are outlined by hardware components and labeled in the block diagram. The fifth subsystem is the phone app. The purpose, interconnections, and requirements of each subsystem are described in detail through this section.

\begin{figure}[!h]
	\centering
	\includegraphics[width=0.95\textwidth]{Full_Block_Diagram_v6.png}
	\caption{Block Diagram}
	\label{fig:block}
\end{figure}

% Each subsystem needs detailed purpose and description, table of parts used for it, circuit schematics, simulations, calculations, coding flow charts, etc.
% Each subsystem also needs an Requirements & Verification table
\subsection{Power Subsystem}

The power subsystem is made up of a rechargeable battery, a battery management system capable of overcharge voltage, overdischarge voltage and overcurrent detection, a linear voltage regulator, a boost converter, and a buck converter for charging. This subsystem is responsible for providing power to all other hardware components within the system in a safe manner. Without this subsystem, none of the main requirements could be met as each operation of the system requires power. The power subsystem connects to the other three hardware subsystems (the card reader, display, and communication subsystems) via a 3.3V DC voltage used for logic powering provided by the voltage regulator. The power subsystem connects to the display subsystem through an additional 9.6V DC voltage used for the LED backlighting of the LCD display.

This linear voltage regultor is a low-dropout linear regulator used to convert from the 3.0V-3.7V input from the battery management system to the 3.3V output (when the input is lower than 3.3V, the output follows it directly). This type of regulator was chosen as it is good for dropping small voltage levels (such as from 3.7V to 3.3V) and has a much lower noise and EMI profile than switching converters which is ideal for operation on a board with high frequency digital operations and RF traces. The LDO voltage regulators tend to have lower efficiencies than switching converters, so to combat these effects, an existing adjustable LDO was utilized. The biasing and protection capacitors included with this component are shown in Figure \ref{fig:LDO_Schem.png}. The most important biasing aspect of this block are the feedback resistors. The resistor value calculation sets $R_1 = 10k\Omega$ and calculates
\[ R_2 = R_1 (\frac{V_out}{1.216} - 1) = 17.14k\Omega.\]
This value will not be acheivable with a single component so two $33k\Omega$ resistors are used in parallel along with a $2k\Omega$ potentiometer to allow for fine tuning of the output voltage to the desired value. Another reason for choosing the LDO is that once the battery voltage drops below 3.3V, the output voltage will simply follow the input voltage. This lower voltage is acceptable to all of the components requiring this power bus and as such is allowed for power usage. The battery management system then cuts the output off at 3.0V.

\begin{figure}[!h]
	\centering
	\includegraphics[width=0.65\textwidth]{LDO_Schem.png}
	\caption{Power Subsystem LDO Linear Voltage Regulator Schematic}
	\label{fig:boost_schem}
\end{figure}

The boost converter is a switching converter used to take the 3.0-3.7V input from the battery management system and produce a 9.6V output. The boost converter block is shown in Figure \ref{fig:boost_schem}. This boost converter requires a PWM signal to control a MOSFET which enables and disables current from the input to the output. The boost converter functions by storing power within the capacitor and inductor bank and transfering this power at a rate defined by the duty cycle of the PWM signal to the MOSFET. A small inductor value for this circuit leads to higher stability but greater current fluctations. A lower inductor value increases the efficiency with less stability. The chosen inductor is a midpoint of the suggestions proposed in \cite{boost}. The feedback biasing is also implemented to be adjustable. Here $R_{FB1}$ is set at $82k\Omega$ and $R_{FB1}$ is calculated to be $9.5k\Omega$ implemented with a $8.2k\Omega$ resistor and a $2k\Omega$ potentiometer. The buck converter is a switching converter use to take the 5.0V USB input to the 4.2V charging voltage of the battery. The buck converter is shown in Figure \ref{fig:buck_schem}. A buck converter operates very similarly to a boost converter running in reverse. This module includes a second MOSFET for switching to avoid unintended floating effects. This module also is designed to be tuned using the included potentiometer.

\begin{figure}[!h]
	\centering
	\includegraphics[width=0.95\textwidth]{Power_Boost_Schem.png}
	\caption{Power Subsystem Boost Converter Schematic}
	\label{fig:boost_schem}
\end{figure}

\begin{figure}[!h]
	\centering
	\includegraphics[width=0.95\textwidth]{Power_Buck_Schem.png}
	\caption{Power Subsystem Buck Converter Schematic}
	\label{fig:buck_schem}
\end{figure}

The rechargeable battery will be a lithium ion battery. This is a popular rechargeable battery that tends to have a good energy and power density. The main battery draw components are the MCU module (500mA max), the RFID reader (150mA max), and the display (210 mA max) leading to a total current draw of 960mA. This battery has a rated output of 1.5A continuous within this range. The lithium ion battery output will be connected to a battery protection IC. The battery protection is used to monitor the status of the battery in terms of output voltage and current draw. This protects the battery from being over or undercharged and from being overdrawn by being able to cut power connections. This allows for a longer battery life and a safer device. The battery protection circuit is shown in Figure \ref{fig:prot_schem}

\begin{figure}[!h]
	\centering
	\includegraphics[width=0.65\textwidth]{BatteryProtection-Schematic.png}
	\caption{Power Subsystem Battery Protection Schematic}
	\label{fig:prot_schem}
\end{figure}

The power subsystem has a set of basic requirements that ensure its proper operation. These requirements are as shown in Table \ref{tab:p}

\begin{table}[!h]
	\caption{Power Subsystem Requirements and Verification}
	\label{tab:p}
	\centering
	{\small
	\begin{tabular}{| p{0.45\linewidth} | p{0.45\linewidth} |}
 		\hline
 		\textbf{Requirements} & \textbf{Verification} \\ 
 		\hline
 		\begin{enumerate}
 			\item When the voltage of the lithium ion battery cell exceeds the overcharge protection voltage of 4.25V (within a tolerance of $\pm$0.05V), the battery protection circuit must be able to disconnect the connected components and inhibit charging by turning off the charge control MOSFET. 
		\end{enumerate} & \begin{enumerate}[label=\alph*)]
 			\item To verify the correction functionality of the overcharge battery protection circuit, we can replace the lithium ion battery cell with a ramp DC voltage source, starting from the nominal 3.7V battery voltage and slowly ``ramping up'' the voltage by 0.1V increments to the rated 4.25V overcharge protection voltage. During this test, we can probe the gate (voltage measurement), drain and source (current measurement) terminals using an oscilloscope to confirm that the charge control MOSFET enters the cutoff region, as intended. Further, we must also probe the load of this circuit and take a resistance measurement between the two output terminals to confirm there is a very high impedance (i.e. the load is disconnected from the circuitry).
		\end{enumerate} \\
		\hline
		\begin{enumerate}
		\setcounter{enumi}{1}
 			\item When the voltage of the lithium ion battery cell falls below the overdischarge protection voltage of 2.40V (within a tolerance of $\pm$0.05V), the battery protection circuit must be able to disconnect the connected components and inhibit discharging by turning off the discharge control MOSFET.
		\end{enumerate} & \begin{enumerate}[label=\alph*)]
 			\item To verify the correction functionality of the overdischarge battery protection circuit, we can replace the lithium ion battery cell with a ramp DC voltage source, starting from the nominal 3.7V battery voltage and slowly ``ramping down'' the voltage by 0.1V increments to the rated 2.40V overdischarge protection voltage. During this test, we can probe the gate (voltage measurement), drain and source (current measurement) terminals using an oscilloscope to confirm that the discharge control MOSFET enters the cutoff region, as intended. Further, we must also probe the load of this circuit and take a resistance measurement between the two output terminals to confirm there is a very high impedance (i.e. the load is disconnected from the circuitry).
		\end{enumerate} \\
		\hline
	\end{tabular}
	}
\end{table}

\begin{table}[!h]
	\caption{Power Subsystem Requirements and Verification Continued}
	\label{tab:p_cont}
	\centering
	{\small
	\begin{tabular}{| p{0.45\linewidth} | p{0.45\linewidth} |} 
 		\hline
 		\textbf{Requirements} & \textbf{Verification} \\ 
		\hline
		\begin{enumerate}
		\setcounter{enumi}{2}
 			\item The voltage regulator circuit must be able to maintain a voltage within a safe operating range of 3.3V $\pm$ 0.1V (safe operating range for average CMOS device \cite{TI_inverter}). If the input voltage falls below this range, the voltage regulator circuit must then match (follow) the input voltage signal to the LDO regulator (< 3.3V). Otherwise, the LDO regulator must output a constant voltage 3.3V
		\end{enumerate} & 
			To verify the correction functionality of the voltage regulator circuit, we can, again, test the operation of the circuitry under various conditions, which are as follows: (1) variations in the input voltage signal, variations in the load conditions.
			\begin{enumerate}[label=\alph*)]
			\item Using a waveform generator, we can apply a step-wave that varies between 3V and 3.7V, with a DC offset of 3.3V. Then, we can simply probe the output voltage of the LDO regulator to confirm that when the input voltage signal falls below 3.3V, the output will follow the input voltage signal. Any voltage $\geq$ 3.3V, MUST output 3.3V. 
			\item The MCU is rated to have a maximum current draw of 500mA. Therefore, while keeping the input voltage constant at 3.7V (output from the battery protection circuit), we can connect the LDO regulator to a load power supply and induce variable current draw (from 100mA to 500mA) and confirm that regardless of variations in current draw, the output voltage remains as intended.
		\end{enumerate} \\
		\hline
		\begin{enumerate}
		\setcounter{enumi}{3}
 			\item The boost converter circuit must step up the lithium ion battery voltage of 3.7V to a DC voltage of 9.6V ($\pm$ 0.2V) as to provide enough power to the LED backlighting of the LCD display.
		\end{enumerate} & \begin{enumerate}[label=\alph*)]
 			\item To verify the correction functionality of the boost converter circuit, we can probe the VIN and VOUT terminals using 2 channels on an oscilloscope. Then, we can confirm that the input and output voltages are at the intended values of 3.7V and 9.6V, respectively.
		\end{enumerate} \\
		\hline
		\begin{enumerate}
		\setcounter{enumi}{4}
 			\item The buck converter circuit must step down the USB charging port DC voltage from 5V to 4.2V $\pm$ (0.1V).
		\end{enumerate} & \begin{enumerate}[label=\alph*)]
 			\item To verify the correction functionality of the buck converter circuit, we can probe the VIN and VOUT terminals using 2 channels on an oscilloscope. Then, we can confirm that the input and output voltages are at the intended values of 5V and 4.2V, respectively.
		\end{enumerate} \\
		\hline
	\end{tabular}
	}
\end{table}

\subsection{Communication Subsystem}

The communication subsystem is made up of the Bluetooth transceiver, 2.4GHz antenna and the microcontroller. These components are combined into a single module which can be used as the central microcontroller within the device. The reasoning for using a single module is explained next. This subsystem is responsible for communicating the card and game data from the central device to the phone app as well as receiving information from the phone app. The subsystem is vital for the second and third main requirements in sending the game data and allowing the use of this data within the phone app. The communication subsystem connects to the power subsystem through the 3.3V DC supply provided by the voltage regulator of the power system, to the card reader subsystem through the transfer of card data through an SPI interface to the microcontroller, and to the phone app through a 2.4GHz Bluetooth v4.2 wireless signal provided by the internal antenna.

\begin{figure}[h!]
	\centering
	\includegraphics[width=0.65\textwidth]{Comms_Schem.png}
	\caption{Communication Subsystem Schematic}
	\label{fig:comm_schem}
\end{figure}

This subsystem has been adjusted from previous plans to be consolidated within a single purchasable component. The component chosen for this purpose is the ESP32-WROVER-E module (henceforth referred to as the MCU module) which combines a microcontroller, flash memory, DRAM, and 2.4GHz antenna into a single surface-mount component. This simplifies the design of the communications subsystem by allowing the Bluetooth packets and connections to be initiated by the built-in Bluetooth protocol and API of the microcontroller, rather than communicated through a serial communication protocol to a new chip to do the same operations. This also prevents the design of high frequency RF traces and discrete matching networks on our PCB to connect the Bluetooth transceiver component to an external antenna. Traces at such a frequency can be heavily affected by tolerances of the PCB manufacturing leading to poor performance as compared to a combined module with guaranteed operation specifications as chosen here. Lastly, the combined module is well documented allowing for a faster development process in getting a functioning connection between this device and a phone app allowing for more focus on other hardware components.

The MCU module is shared by other subsystems requiring embedded C code to function properly and as such the design requirements and reasoning for this device are split among the subsystem descriptions. For the communication subsystem, the MCU module provides an API for generating Bluetooth v4.2, a low power alternative to classic Bluetooth ideal for battery powered devices with a low required throughput such as this device. As previously mentioned, the MCU module provides a 2.4GHz antenna to transmit the Bluetooth packets making this module an ideal choice. The module also has an associated low-cost development kit (ESP-DEVKITC-VE) allowing for easy coding and troubleshooting before programming to our finished board.

\begin{figure}[!h]
	\centering
	\includegraphics[width=0.9\textwidth]{reset_debounce.png}
	\caption{Reset Button Debounce Simulations}
	\label{fig:reset_debounce}
\end{figure}

The communication subsystem is included within the schematic shown in Figure \ref{fig:comm_schem}. The subsystem does not require an individual schematic as it is solely comprised of the MCU module and programming circuitry, and the MCU module is shared between multiple subsystems. The reset button shown is used to boot the device. A simple RC circuit with a time constant of $\tau = RC = 1ms$ is applied after the push button to provide debouncing and avoid multiple resets from a single press. The simulation of this reset button being pressed is shown below in Figure \ref{fig:reset_debounce}. The boot switch is a programming feature as this switch controls if the microcontroller downloads a program through the UART interface on boot (IO0 = low), or if the microcontroller uses the saved program from the flash memory for its instructions (IO0 = high). The boot control also relies on the IO2 pin being low on boot and has been connected to a pull-down resistor for this reason. Simple breakout pins will be included (as shown in the schematic) to connect to a UART interface for programming the microcontroller.

The communication module requires significant software control. The firmware loaded to the MCU will have several main functions pertaining to the different subsystems. The communication functions include creating bluetooth packets and using the ESP32 API to send the packets to a connected device as well as checking for any received packets from a device. The flow chart showing these operations is in Figure \ref{fig:MCU_flow}

\begin{figure}[!h]
	\centering
	\includegraphics[width=0.9\textwidth]{MCU_Flow.png}
	\caption{MCU Firmware Flow Chart}
	\label{fig:MCU_flow}
\end{figure}

The requirements and validation methods for the communication subsystem that ensure its proper operation are shown in Table \ref{tab:comms_rv} Notice that there are no requirements for latency. As this device is sending very low amounts of information (such as card IDs which can be transmitted in a single byte) and the actions are not time sensitive, latency is not included as a requirement. The throughput requirements are very low for Bluetooth standards. This is again to account for the low data transfer required and to give space in our firmware and hardware components in the case they are less efficient than state-of-the-art designs.

\begin{table}[!h]
	\caption{Communication Subsystem Requirements and Verification}
	\label{tab:comms_rv}
	\centering
	\begin{tabular}{| p{0.45\linewidth} | p{0.45\linewidth} |} 
 		\hline
 		\textbf{Requirements} & \textbf{Verification} \\ 
 		\hline
 		\begin{enumerate}
 			\item The communication subsystem must use Bluetooth v4.2 \cite{IEEE_bt} to transmit and receive information to and from the phone app at a distance of 3ft with an average throughput of 5kbps and a bit error rate of less than 1\%.
		\end{enumerate} & \begin{enumerate}[label=\alph*)]
 			\item The MCU Firmware will be loaded onto the assembled device using the UART connection. A basic function will be implemented to create a 5kb binary file holding pseudo-random bytes in a deterministic manner. The MCU will use the ESP32 Bluetooth API to transmit packets and receive acknowledgements.
 			\item The central PCB will be placed 3ft away from a device containing the phone app. The phone app functionality is not important in this test, it simply must be capable of receiving Bluetooth packets, saving the information, and sending acknowledgements. The same data generation function will be provided to the phone app to be able to check for errors.
 			\item The full 5kb file will be transmitted and the total time will be recorded to obtain an average throughput measure. The Bluetooth protocol often accounts for many bit errors through correction schemes, but the generated file on the phone app will be used to check bit error rate.
		\end{enumerate} \\
 		\hline
	\end{tabular}
\end{table}

\subsection{Card Reader Subsystem}

The card reader subsystem is made up of the microcontroller, the RFID sensor, and the antenna. The subsystem is responsible for reading the card ID of each card swiped past the device. The subsystem is vital for the first main requirement of sensing and distinguishing each of the 52 cards. The card reader subsystem connects to the power subsystem through the provided $3.3V_{DC}$ supply, and to the communication subsystem through the passing of card data through an 8-bit parallel interface to the microcontroller.

The RFID sensor generates a 13.56MHz frequency signal that powers the RFID tags on the cards and results in the cards transmitting a signal containing the serial ID of the tag. This tag data reception is controled using the ISO 14443A RFID protocol. The microcontroller will communicate with the reader to receive the serial ID. Additionally, the card reader IC requires a matching circuit to correct against return power transfer from the antenna to the reader during transmission. The schematic for this susbsystem is shown in Figure \ref{fig:card_schem}

\begin{figure}[!h]
	\centering
	\includegraphics[width=0.7\textwidth]{unknown.png}
	\caption{Card Reader Subsystem Schematic}
	\label{fig:card_schem}
\end{figure}

The basic requirements for the card reader subsystem that ensure its proper operation are shown in Table \ref{tab:card_rv}.

\begin{table}[!h]
	\caption{Card Reader Subsystem Requirements and Verification}
	\label{tab:card_rv}
	\centering
	\begin{tabular}{| p{0.45\linewidth} | p{0.45\linewidth} |} 
 		\hline
 		\textbf{Requirements} & \textbf{Verification} \\ 
 		\hline
 		\begin{enumerate}
 			\item Tags must reliably be able to be read quickly and accurately, independent of the RFID Reader IC/Antenna portion of the module.
		\end{enumerate} & \begin{enumerate}[label=\alph*)]
 			\item Either embed the tags in the middle of the card or stick the tags on the outside without severely affecting the overall thickness and feel of the cards. We have chosen tags that will be thin and flexible enough to be almost unnoticeable from a different deck's dimensions. Each deck will be stress-tested with multiple shuffles and leafings.
 			\item Phone apps are able to pick up these RFID tags, courtesy of Avery-Dennison (the company selling the RFID tags) and we will ensure that these tags work properly upon purchase and delivery.
		\end{enumerate} \\
		\hline
		\begin{enumerate}
		\setcounter{enumi}{1}
 			\item Reliably pick up 13.56MHz signals from <17mm distance with little to no reflection
		\end{enumerate} & \begin{enumerate}[label=\alph*)]
 			\item Measure the antenna VSWR using a spectrum analyzer, signal generator, and directional coupler to find out how much energy sent to the antenna is reflected back. The s-parameter simulation will give an idea of how well the antenna is impedance matched to the transmission line. A VSWR under 2 will be suitable for this application.
		\end{enumerate} \\
		\hline
		\begin{enumerate}
		\setcounter{enumi}{2}
 			\item Ensure the internal impedances of the RFID Reader IC (which can be set between 1 and 100$\Omega$) and the RF Antenna (50 or 80 $\Omega$) will have matched loads along the transmission line.
		\end{enumerate} & \begin{enumerate}[label=\alph*)]
 			\item Run simulations using ADS plugging in terminals at both ends representing the antenna and IC and with the lumped components in between. Tolerances of 10\% and 5\% will be tested to determine the quality of components we will need.
		\end{enumerate} \\
		\hline
		\begin{enumerate}
		\setcounter{enumi}{3}
 			\item Properly transmit RFID tag info to MCU at Pin 5 at 13.56MHz.
		\end{enumerate} & \begin{enumerate}[label=\alph*)]
 			\item Probe pin 5 to an oscilloscope and scan RFID tag for the reader to pick up. Note the tag scanned and recognize its reading value on the oscilloscope at 13.56MHz.
		\end{enumerate} \\
		\hline
	\end{tabular}
\end{table}

\subsection{Display Subsystem}

The display subsystem is made up of the microcontroller and the display. The subsystem is responsible for showing the user the game actions and current game information. The subsystem is vital for the second main requirement in displaying to the players and dealer the game status and next actions. The game status display subsystem connects to the power subsystem through the 3.3V DC logic supply and 9.6V DC LED backlight supply, and the card reader subsystem through the transfer of card data via SPI interface to the microcontroller.

The LCD display within the subsystem is an output device showing the video data provided to it by the microcontroller. The LCD display chosen receives input data through a parallel 24bit RGB input. This input method requires a large number of GPIOs from the microcontroller which is not convenient for any hand-soldering based projects\footnote{Many microcontrollers with a large number of GPIOs come in BGA (Ball Grid Array) packaging or QFP/QFN (Quad Flat Packaging with or without leads) packaging with a very small pin pitch. These packaging formats are very hard to solder with only a hand soldering iron.}. Alternative displays with fewer pins exist that use protocols such as MIPI DSI, but this protocol is not publicly available and requires significant work to design a functioning interface. To handle this issue, high frequency shift registers are used for each color channel. The shift register has a serial input and 8-bit parallel output allowing for the use of three GPIOs to define each pixel color instead of 24 given the microcontroller can generate output signals at a rate at least eight times the specified clock signal.

\begin{figure}[!h]
	\centering
	\includegraphics[width=0.6\textwidth]{Display_Schem.png}
	\caption{Display Subsystem Schematic}
	\label{fig:display_schem}
\end{figure}

The chosen components meet these design requirements as the $320\times240$ pixel LCD specifies a typical data clock frequency of $6.5MHz$. This screen size requires a $230.4kB$ of memory for a full screen buffer. The MCU module can utilize clocks up to $160MHz$ for LCD control (much larger than eight times the LCD clock requirement) and can output up to three clock signals allowing for the control of both the LCD data clock ($6.5MHz$) and faster shift register clocks ($52MHz$). The chosen MCU module also has a large amount of included memory external to the microcontroller. An $8MB$ PSRAM unit is included within the module providing plenty of space for frame buffers. This memory can be accessed at a rate of up to $80MHz$ with up to 32-bit reads and writes. This data size and access rate far exceeds the requirements set by the LCD allowing for efficient and easy writing reading from a potential double-buffered display. An additional level of redundancy is added with included internal memory components. The MCU includes 520kB of internal SRAM to be used in the case of the external PSRAM unit failing to meet the application needs for any reason.

The connections between the LCD, shift registers, and MCU module are shown in the schematic in Figure \ref{fig:display_schem}. The LCD is controlled by firmware through RGB data output pins and clock outputs and thus is included within the flow diagram .

The requirements and verification methods for the display subsystem that ensure its proper operation are shown below in Table \ref{tab:display_rv} and \ref{tab:display_rv_cont}

\begin{table}[!h]
	\caption{Display Subsystem Requirements and Verification}
	\label{tab:display_rv}
	\centering
	\begin{tabular}{| p{0.45\linewidth} | p{0.45\linewidth} |} 
 		\hline
 		\textbf{Requirements} & \textbf{Verification} \\ 
 		\hline
 		\begin{enumerate}
 			\item The microcontroller must generate both 6.5MHz ($\pm 0.1MHz$) and 52MHz ($\pm 1MHz$) clock signals simultaneously through its clock output pins.
		\end{enumerate} & \begin{enumerate}[label=\alph*)]
 			\item Connect the ESP32-WROVER-E module to a 3.3V power supply. 
 			\item Program the MCU in download mode using the UART interface and set the GPIO control registers to have CLK\_OUT2 at pin 35 and CLK\_OUT3 at pin 34 with frequencies 52MHz and 6.5MHz respectively. 
 			\item Connect these pins to separate channels of an oscilloscope. Use the frequency analysis features of the oscilloscope to record the frequency of each output signal.
		\end{enumerate} \\
		\hline
		\begin{enumerate}
		\setcounter{enumi}{1}
 			\item The microcontroller must output the screen data serially to the three specified GPIO pins with a clock rate of 52MHz ($\pm 1MHz$).
		\end{enumerate} & \begin{enumerate}[label=\alph*)]
 			\item Connect the ESP32-WROVER-E module to a 3.3V power supply.
 			\item Program the MCU in download mode using the UART interface and set the GPIO control registers so pins 6,7, and 8 are the RGB output pins respectively. Provide a sample 8-bit single channel color value (xAA) to be sent serially to the output pins at a rate of 52MHz.
 			\item Connect the output pins to separate channels of an oscilloscope. The selected sample value requires the changing of pin value at each clock period. Use the frequency analysis features of the oscilloscope to record the frequency of each output signal to ensure 52MHz operations.
		\end{enumerate} \\
		\hline
	\end{tabular}
\end{table}

\begin{table}[!h]
	\caption{Display Subsystem Requirements and Verification Continued}
	\label{tab:display_rv_cont}
	\centering
	\begin{tabular}{| p{0.45\linewidth} | p{0.45\linewidth} |} 
 		\hline
 		\textbf{Requirements} & \textbf{Verification} \\ 
 		\hline
		\begin{enumerate}
		\setcounter{enumi}{2}
 			\item The microcontroller must write to external PSRAM to update displayed screen data while display is active.
		\end{enumerate} & \begin{enumerate}[label=\alph*)]
 			\item With device fully constructed (battery connected, display connected). Program the MCU in download mode using the UART interface to contain a single buffer in PSRAM full of a single background tone. Create a function to move a contrasting square to different positions in the buffer.
 			\item Run the device using the PSRAM as the memory input to the display. Call the function repeatedly during operation. Observe the square on the display to ensure intended output matching the buffer location in the PSRAM.
		\end{enumerate} \\
 		\hline
	\end{tabular}
\end{table}

\subsection{Phone App Subsystem}

The phone app subsystem is made up of solely the application used by the phone. The subsystem is responsible for receiving the communications from the central device and sending control information back to the device. The subsystem is vital for the third main requirement in the receipt of game data by the phone and display of optimal moves or game statistics. The phone app subsystem connects only to the communication subsystem through the Bluetooth transmission of game information.

The phone app relies on the device having Bluetooth capabilities. Most modern devices have such capabilities so this should not be a difficult requirement to meet. Android devices are slightly easier to develop with since Android studio and development information is open to the public, therefore the phone app will be developed initially for an Android device using Android Studio.

\begin{table}[!h]
	\caption{Phone App Subsystem Requirements and Verification}
	\label{tab:app_rv}
	\centering
	\begin{tabular}{| p{0.45\linewidth} | p{0.45\linewidth} |} 
 		\hline
 		\textbf{Requirements} & \textbf{Verification} \\ 
 		\hline
 		\begin{enumerate}
 			\item The app will utilize the Bluetooth capabilities of the device to send and receive data with the central device.
		\end{enumerate} & \begin{enumerate}[label=\alph*)]
 			\item A sample ``Hello World!'' packet will be created and sent to the central device. The central device will respond with an acknowledgement. The phone app must be able to receive this acknowledgement ensuring both transmit and receive operations are functional.
		\end{enumerate} \\
		\hline
		\begin{enumerate}
		\setcounter{enumi}{1}
 			\item The app will use card information to accurately calculate the ``strength'' measure or the percent of possible alternate hands against which the user's hand will win.
		\end{enumerate} & \begin{enumerate}[label=\alph*)]
 			\item A random game setup will be created with two cards provided to the user and a total of five cards available to all players. The algorithm will be tested by providing it this information and the output strength measure will be generated using the first three commonly available cards, then the first four, and finally all the commonly available cards (similar to how a Texas Hold'em game would progress). The algorithm will log all available hands at each stage and the percentile of the user's hand. These outputs will be verified manually by the testers to ensure proper operation. 
		\end{enumerate} \\
		\hline
		\begin{enumerate}
		\setcounter{enumi}{2}
 			\item The app will display the measures and suggestions in the form of text for the user.
		\end{enumerate} & \begin{enumerate}[label=\alph*)]
 			\item The software will combine the Bluetooth and strength measure calculations into a working app. The app will be tested in a game situation and card IDs will be sent from the central device. The app must display this information in an understandable manner to an average user (someone not familiar with the design of the device).
		\end{enumerate} \\
 		\hline
	\end{tabular}
\end{table}

The app receives game data from the communications system through Bluetooth packets. The app then uses the information of the cards within the player's hand and all other cards showing on the board and uses to determine the strength of the users hand in relation to other possible hands in the current scenario. The strength measurement is calculated as the hand percentile indicating the percentage of currently possible hands against which the user's hand will win. The app displays this strength measurement to the user and will use it to suggest moves to the user. As a baseline, this suggestion will be implemented off a measurement threshold where higher strength measurements lead to more aggressive moves (betting) and lower measurements lead to less aggressive moves (folding). With enough time and successful work, the app could potentially use a deep network model to continue to provide better suggestions with more use.

The app will also be able to provide control information to the central device. The phone app will set the game type and the number of players within the game. This information is important to the functionality of the display module to provide the correct actions of for the dealer/players. The phone app will also be able to reset the hand in the case of a dealing issue. The operations of the app are shown in the flow chart in Figure \ref{fig:app_flow}.

\begin{figure}[!h]
	\centering
	\includegraphics[width=0.4\textwidth]{app_flow.png}
	\caption{App Flow Chart}
	\label{fig:app_flow}
\end{figure}

The basic requirements for the phone app subsystem that ensure its proper operation are shown in Table \ref{tab:app_rv}. As can be seen, the Bluetooth capabilities do not have the same strict requirements as specified in the communications subsystem. The phone hardware and bluetooth protocol are much more optimized than our custom built device leading to a much higher expected performance if viewed individually. Within our system, the central device will be the bottleneck for bluetooth communication rates and thus are only measured in the communications subsystem.

% Tolerance analysis for the hardest part of the design (most likely to fail or cause problems)
\subsection{Tolerance Analysis}

The success of the Odds Booster relies heavily on the ease and accuracy of cards being dealt to the players. Our project depends on the usage of Radio-Frequency Identification technology to correctly ascertain the card that is being dealt to a specific player at the table. Not only is this supposed to be easy, it's supposed to be reliable. With any misreading or failure to transmit, the player is lost and will have to reshuffle and redeal the hand. From both a manufacturing and physical realm, the use of RF technology/trace tapering and matching networks is the most critical part of the project.

First and foremost, the playing cards must have an RFID NFC tag on each and every one of them, able to communicate with our card-reader module to eventually send to our central MCU and Bluetooth unit and finally transmit the data to the phone app in the end. The card-reader module requires precise circuitry to ensure preserved signal integrity and speed; an antenna, matching network, RFID IC with an oscillator, and potentially the usage of RF trace tapering. Figure \ref{fig:taper} shows an example of a taper increasing impedance by decreasing the trace width, as well as showcasing the quality needed for a smooth taper.

\begin{figure}[!h]
	\centering
	\includegraphics[width=0.6\textwidth]{image6.png}
	\caption{RF Taper Trace Example}
	\label{fig:taper}
\end{figure}

The region in between the antenna and the RFID Reader IC must account for the difference in intrinsic impedances between the two devices. This region will be where we will match the impedances, whether it be through trace tapering, a lumped-component matching network, or a combination of the two. Trace impedance ($Z_o$) is a critical factor in the effort to control reflections/dampening in our RF signal; however, in traces shorter than 1/20th of a wavelength long, impedance matching is usually not important. 

At 13.56MHz, we have our 1/20th wavelength measurement at around 1.1m and we are not intending to have a transmission line of that length in our PCB. Lumped components will likely be used to exactly match our impedance levels, since our transmission line is relatively tiny at the frequency we are operating at. 

Figure \ref{fig:reader} is taken from the RFID Reader IC that showcases an application example circuit diagram with a directly matched antenna, and Figure \ref{fig:antenna} is taken from an active antenna concept for the MFRC500 (the RFID Reader IC we've chosen) with the correct lumped component values for the receiving circuitry and matching circuitry. Since the transmission line isn't long enough to have serious concern for mismatched loads, we can assume a 10\% tolerance on these components will preserve signal integrity.

\begin{figure}[!h]
	\centering
	\includegraphics[width=0.6\textwidth]{image7.png}
	\caption{RF Reader IC Typical Application}
	\label{fig:reader}
\end{figure}

\begin{figure}[!h]
	\centering
	\includegraphics[width=0.6\textwidth]{image3.png}
	\caption{Antenna Application Circuit Example}
	\label{fig:antenna}
\end{figure}

In the datasheet for the RFID Reader IC we have chosen, Section 9.8 covers the Oscillator circuit in which they explicitly remark that an external clock source is not recommended and that an internal oscillator buffer is provided to complement the minimization of clock jitter. 

Calculating jitter from phase noise is highly dependent on the quality of the oscillator. In Figure \ref{fig:noise}, the oscillator frequency is chosen to be 100MHz for discussion purposes, and the graph shows the phase noise curve is approximated by several tranches indicating different data points relevant to the calculation of jitter.

\begin{figure}[!h]
	\centering
	\includegraphics[width=0.6\textwidth]{image4.png}
	\caption{Phase Noise vs Frequency Offset for 100MHz Oscillation}
	\label{fig:noise}
\end{figure}

\section{Cost and Schedule}

\subsection{Cost Analysis}

The typical graduate from ECE at Illinois typically makes around \$35/hour, so that is the hourly pay we will roll with. For 12 weeks at around 15 hours per week, with the 2.5x factor, we arrive at a total of \$15,750 per person. With three people in our group, the total employee cost is \$47,250. All necessary parts are included within Table \ref{tab:block_comps} and \ref{tab:block_comps_cont}. The communication subsystem will total up to \$33.15, the card reader subsystem will total up to \$17.50, and the display subsystem will total up to \$31.01. In total, the parts add up to \$81.66
Total sum: $\$47,250+\$81.66 = \$47,331.66$.

% table holds any purchasble parts and (if they serve as a block within the block diagram) the block label they correspond with
\begin{table}[!h]
	\caption{Bill of Materials}
	\label{tab:block_comps}
	\centering
	{\small
	\begin{tabular}{ |c|c|c|c| } 
 		\hline
 		\textbf{Part} & \textbf{Component Name} & \textbf{Quantity} & \textbf{Total Cost} ($\$$) \\
 		\hline
 		\hline
 		\multicolumn{4}{|c|}{Power Subsystem} \\
 		\hline
 		Battery & Mikro-4474 & 1 & 14.50 \\
 		Battery Connector & B2B-XH-A(LF)(SN) & 2 & 0.30 \\
 		USB Micro-B Port & GCT USB3505-30-A-KIT & 1 & 2.81 \\
 		On-Off-On Switch & Nidec Copal 8SS1022-Z & 1 & 4.17 \\
 		Header Pins & Wurth 61300511121 & 1 & 0.26 \\
 		Black Power Wire & LAPP 4160401 & 1 & 0.70 \\
 		Red Power Wire & LAPP 4160404 & 1 & 0.70 \\
 		Boost Gate Driver & ABLIC S-8337ABIA-T8T1U & 1 & 1.52 \\
 		10$\mu$ Inductor & Bourns SRP0415-100K & 1 & 1.12 \\
 		Diode & Panasonic DB2X20700L & 2 & 0.50 \\
 		N-MOS & Nexperia BSH105,215 & 2 & 0.76 \\
 		280k Resistor & Panasonic ERJ-PB6D2803V & 1 & 0.25 \\
 		240k Resistor & Panasonic ERJ-P06J244V & 1 & 0.13 \\
 		200k Resistor & Panasonic ERJ-P06J204V & 1 & 0.13 \\
 		82k Resistor & Panasonic ERJ-P06J823V & 2 & 0.38 \\
 		8.2k Resistor & Panasonic ERJ-P06J822V & 3 & 0.39 \\
 		2k Potentiometer & Bourns 3361P-1-202GLF & 2 & 2.54 \\
 		180p Capacitor & KEMET C0805C181J3HAC7800 & 2 & 0.22 \\
 		0.1$\mu$ Capacitor & Samsung CL21B104KACNNNC & 6 & 0.60 \\
 		10$\mu$ Capacitor & Samsung CL21A106KOQNNNE & 2 & 0.22 \\
 		Linear Voltage Regulator & TI LP3965ESX-ADJ/NOPB & 1 & 4.62 \\
 		68$\mu$ Capacitor & TDK C3216X5R0J686M160AB & 1 & 0.85 \\
 		33$\mu$ Tantalum Capacitor & Nemco PCT33/10CK & 1 & 0.26 \\
 		100p Capacitor & Samsung CL21C101JBANNNC & 2 & 0.20 \\
 		10k Resistor & Panasonic ERJ-P06J103V & 3 & 0.39 \\
 		33k Resistor & Panasonic ERJ-P06J333V & 3 & 0.39 \\
 		Buck Gate Driver & Microchip MIC2169YMM & 2 & 5.04 \\
 		N-MOS & Diodes DMG2302UQ-13 & 5 & 2.20 \\
 		3.3$\mu$ Inductor & Bourns SRP0415-3R3K & 1 & 1.12 \\
 		100$\mu$ Capacitor & Samsung CL32A107MPVNNNE & 1 & 1.14 \\
 		150$\mu$ Tantalum Capacitor & KEMET T491X157K020AT & 1 & 2.53 \\
 		4.7$\mu$ Capacitor & Samsung CL10A475KP8NNNC & 1 & 0.10 \\
 		150p Capacitor & Samsung CL21B151KBANNNC & 1 & 0.10 \\
 		0.1p Capacitor & Vishay VJ0603D0R1BXPAJ & 1 & 0.45 \\
 		2 Resistor & Panasonic ERJ-6DQJ2R0V & 1 & 0.24 \\
 		560 Resistor & Panasonic ERJ-P06J561V & 1 & 0.13 \\
 		1.5k Resistor & Panasonic ERJ-P06J152V & 1 & 0.13 \\
 		4.7k Resistor & Panasonic ERJ-P06J472V & 1 & 0.13 \\
 		1k Potentiometer & Bourns 3361P-1-102GLF & 1 & 1.27 \\
 		Diode & Surge SL34A & 2 & 0.68 \\
 		100 Resistor & Panasonic ERJ-P06J101V & 1 & 0.13 \\
 		1k Resistor & Panasonic ERJ-P06J102V & 1 & 0.13 \\
 		\hline
	\end{tabular}
	}
\end{table}

\begin{table}[!h]
	\caption{Bill of Materials Continued}
	\label{tab:block_comps_cont}
	\centering
	{\small
	\begin{tabular}{ |c|c|c|c| } 
 		\hline
 		\textbf{Part} & \textbf{Component Name} & \textbf{Quantity} & \textbf{Total Cost} ($\$$) \\
 		\hline
 		\hline
 		\multicolumn{4}{|c|}{Communication Subsystem} \\
 		\hline
 		Microcontroller & ESP32-WROVER-E & 1 & 3.60 \\
 		Development Board & ESP32-DEVKITC-VE & 1 & 11.00 \\
 		USB to UART & Paialu CP2102 & 1 & 8.76 \\
 		On-On Switch & C\&K JS202011SCQN & 1 & 0.55 \\
 		Pushbutton & C\&K PTS636 SL43 SMTR LFS & 2 & 0.24 \\
 		Header Pins & Wurth 61300411121 & 3 & 0.57 \\
 		Jumper Wires 10pc & Mikroe-511 & 1 & 3.00 \\
 		10k Resistor & Panasonic ERJ-P06J103V & 6 & 0.78 \\
 		0.1$\mu$ Capacitor & Samsung CL21B104KACNNNC & 2 & 0.20 \\
 		\hline
 		\multicolumn{4}{|c|}{Card Reader Subsystem} \\
 		\hline
 		RFID Reader & NXP MFRC53101T/0FE,112 & 1 & 24.28 \\
 		13.56MHz Antenna & PulseLarsen Antennas W3102 & 2 & 3.54 \\
 		0.22$\mu$ Capacitor & Samsung CL21B224KAFNNNG & 2 & 0.70 \\
 		13.56MHz Xtal & Wurth 830082070 & 2 & 1.40 \\
 		1$\mu$ Inductor & Bourns SRP0415-1R0K & 3 & 3.36 \\
 		15p Capacitor & KEMET C0805C150K1HAC7800 & 4 & 0.40 \\
 		68p Capacitor & KEMET C0805C680J5GAC7800 & 4 & 0.40 \\
 		0.1$\mu$ Capacitor & Samsung CL21B104KACNNNC & 3 & 0.30 \\
 		8.2p Capacitor & Samsung CL21C8R2DBANNNC & 2 & 0.20 \\
 		4.7u Electrolytic Capacitor & Wurth 865080640004 & 2 & 0.36 \\
 		10k Resistor & Panasonic ERJ-P06J103V & 2 & 0.26 \\
 		820 Resistor & Panasonic ERJ-6ENF8200V & 2 & 0.20 \\
 		2.7k Resistor & Panasonic ERJ-P06J272V & 2 & 0.26 \\
 		\hline
 		\multicolumn{4}{|c|}{Display Subsystem} \\
 		\hline
 		LCD & Orient Display AFY320240A0-3.5N6NTN-R & 1 & 11.53 \\
 		FCP Connector & Hirose FH40-40S-0.5SV & 2 & 5.50 \\
 		Shift Register & Nexperia 74LV164D & 4 & 1.72 \\
 		0.1$\mu$ Capacitor & Samsung CL21B104KACNNNC & 4 & 0.40 \\
 		\hline
	\end{tabular}
	}
\end{table}

\subsection{Schedule}

A complete schedule showing the breakdown of work by week and group member as well as important deadlines is shown in Appendix A.

\section{Ethics and Safety}

Gambling and wagering money on casino games inherently raises an ethics-eyebrow. Getting involved in this space requires the ability to tread lightly and ensure the IEEE Code of Ethics \cite{IEEE_ethics} is followed to a tee. While our project is aimed at improving players' abilities to perform in the casino, there are measures that need to be accounted for during the development of this project and issues that could potentially arise from accidental or intentional misuse of our project. 

First, we must acknowledge what are the common laws that are broken when cheating in the casino. NRS 465.083 \cite{NRS}, according to Nevada law, prohibits players from cheating at a casino game, which by definition is the manipulation of the outcome of the game or the payments made. This can lead to a felony charge, with 1 to 5 years in prison, restitution, and up to \$10,000 in fines. Essentially, this is a fraud charge specifically citing it happening in the casino. The legal definition is stated as ``alter[ing] the elements of chance, method of selection, or criteria which determine'' the results, amount of payment, or frequency of payment in a game \cite{NRS}; This relates to IEEE Code of Ethics I.4 \cite{IEEE_ethics}. In order to avoid ethical breaches, we are requiring that the usage of our project will have all casino cards scanned and in need of compliance with the dealer. Any dealer that is sitting at a table game in the casino will clearly not agree to use our project as it's in clear violation of casino policy, and even with a dealer that is in cahoots with the player will immediately be caught with the ``Eyes in the Sky'' which is the term for the security measures in the casino. Essentially, we are taking the route that there is no secretive way to use this device in the casino that you wouldn't get caught trying to use outside resources to help boost your odds and chances of winning. A custom deck of cards and a table-top card scanner is impossible to get by the security systems at the casino. 

Next, we look into the potential harm that this project can deliver to the players and general public. This relates to IEEE Code of Ethics I.1 \cite{IEEE_ethics}, and due to our planned usage of RFID and Bluetooth technologies, we are not transmitting or receiving any harmful signals that are outside the typical broadband spectrum of frequencies. There will be no cause of concern for the wireless communications occurring throughout the usage of this project. According to the FDA, there is no evidence of any adverse effects associated with the usage of passive RFID \cite{FDA_RFID}.

Furthermore, IEEE Code of Ethics I.3 \cite{IEEE_ethics} references conflicts of interest, which could potentially pertain to the casinos and their disliking to players increasing their odds of winning. Since this is purely an educational tool, there should be no issues with the casinos and the usage of this in the casino proper is virtually impossible thanks to our design of the project and tight security measures that are relevant to every casino.

Since our project does not in any way put others down or engage in any sort of harassment or discrimination, the IEEE Code of Ethics II \cite{IEEE_ethics} is fulfilled as well. 

Lastly, taking a quick look at the Bluetooth Code of Conduct \cite{BT_conduct} will show that we are following each item and staying true to the ethics laid out. We are not violating any law, transmitting harmful content or tampering, or sending excessively high volume data transfers or bandwidth consumption.

Ethical breaches are primarily prevented due to the design of our project, which requires the usage of a table-top card scanner and our own custom deck of cards (which obviously can not be used in the casino). Players are unable to use the Odds Booster if the custom deck of cards isn't being used, since it's the way that players are able to know what the best possible choices are to take in the casino games. The card scanner needs to send the information to the phone app to crunch the data, thereby preventing the use of any other cards. 

As with any design project, there will always be safety concerns that need to be acknowledged, assessed, and accounted for. In the following section, these concerns are discussed with respect to their associated hazards and potential solutions to minimize any possibilities of risk.

Typically, the primary point of safety concern for any electrical system design stems from the source of power. The power source is not only responsible for the system's overall functionality, but can also work to destroy necessary hardware components (or subsystems) due to unanticipated variations in power output. To take this into account, our power subsystem consists of a rechargeable battery, a battery management system, and a voltage regulator. As specified in greater detail in the power subsystem section, the battery management system and voltage regulator ensure the system's overall immunity to variations in current and voltage such that each individual component can operate at a safe level. It is important to note that the rechargeable battery being used is a lithium ion battery. As per the ECE 445 Safety Guidelines \cite{445_safety}, ``Any group charging or utilizing certain battery chemistries must read, understand, and follow guidelines for safe battery usage.'' All members of the group have read and agreed to follow the course staff's guidance regarding high capacity batteries \cite{Li_safety} and will complete all necessary safety training and adhere to the guidelines set forth in the aforementioned document. 

Before moving on, it is worth to note that the maximum voltage our system will operate under is approximately 3.3V - 3.6V. Inherently, we will be working with relatively low voltages and currents. This is reassuring to know that while we may not be as subject to the greater (and more dangerous!) risks posed from working with higher voltage equipment, that does not mean we can neglect basic lab and equipment safety policies and procedures. All members of the group have also completed the required lab safety training to gain access to the lab. 

Overall, this project is considered relatively ``safe'' under the context of adhering to proper safety policies and protocols when in the lab. There are no true significant safety concerns to the end user as well. The user will simply operate the device primarily from his/her phone and be isolated from any true risk of electrical or mechanical interaction as the entire system will be housed in a custom CAD-designed enclosure. This enclosure serves a double purpose in that it will shield the user/environment from exposure to the electrical components and vice versa. 

% Bibliography (references in separate file)
\bibliographystyle{IEEEtran}
\bibliography{references.bib}

\newpage
\section{Appendix A: Schedule}
\begin{figure}[!h]
	\centering
	\includegraphics[height=0.93\textheight]{Schedule.png}
\end{figure}

\end{document}
